%
%

%%%%%%%%%%%%%%%%%%%%%%%%%%%%%%%%
\section{%Anhang Ⅱ \\
Relativistische Theorie des nichtsymmetrischen Feldes}
\label{sec:anh-2}
%%%%%%%%%%%%%%%%%%%%%%%%%%%%%%%%

Bevor ich mit dem eigentlichen Gegenstande beginne,
will ich eine allgemeine Betrachtung über die \enquote{Stärke}
von Systemen von Feldgleichungen im allgemeinen
vorausschicken. Diese Betrachtung ist auch unab-
hängig von der besonderen hier dargestellten Theorie
von Interesse .. Für eine tiefere Durchdringung unseres\
Problems ist sie aber beinahe unentbehrlich.

%%%%%%%%%%%%%%%%
\subsection{Über die \enquote{Kompatibilität}
und die \enquote{Stärke} von Systemen von Feldgleichungen}

%%%%%%%%%%%%%%%%

Wenn gewisse Feldvariable gewählt sind sowie ein
System von Feldgleichungen für diese, so bestimmen die
letzteren im allgemeinen das Feld nicht vollständig,
sondern es bleiben gewisse frei wählbare Größen für
eine Lösung der Feldgleichungen. Je weniger solch frei
wählbare Größen von dem System von Feldgleichungen
zugelassen werden, desto "stärker" ist das System. Es
ist klar, daß man in Ermangelung anderer Gesichts-
punkte einem in diesem Sinne stärkeren System gegen-
über einem weniger starken den Vorzug geben wird.
Es ist unser Ziel, für diese Stärke von Gleichungs-
systemen ein Maß zu finden. Es wird sich dabei zeigen,
daß sich ein solches Maß angeben. läßt, das uns sogar
in den .Stand setzt, die Stärke von Systemen mitein-
ander zu vergleichen, deren Feldvariable nach Zahl und
Art voneinander verschieden sind.

%%%%%%%%%%%%%%%%
\subsection{Relativistische Feldtheorie}

%%%%%%%%%%%%%%%%

Allgemeines
Die eigentliche Leistung der (allgemeinen) Relativi-
tätstheorie liegt darin, daß sie die Physik von der N ot-
wendigkeit der Einführung des "Inertialsystems" (bzw.
der Inertialsysteme) befreit hat. Das Unbefriedigende
an diesem Begriff liegt darin: Er ,vählt ohne Begründung
unter allen denkbaren Koordinatensystemen gewisse
Systeme aus. Es wird dann angenommen, daß die Ge-
setze der Physik nur in bezug auf solche Inertialsysteme
gelten (z. B. der Trägheits-Satz und das Gesetz von der
Konstanz der Lichtgeschwindigk-eit). Dadurch wird dem
Raum als solchem :eine Rolle im System der Physik
zuerteilt, die ihn vor den übrigen Elementen der physi-
kalischen Beschreibung auszeichnet: Er wirkt bestim-
mend auf alle Vorgänge, ohne daß diese auf ihn zurück-
wirken; eine solche Theorie ist zwar logisch möglich,
aber andererseits doch recht unbefriedigend. \textsc{Newton}\index{\textsc{Newton}, I.}
hatte diesen Mangel deutlich empfunden, aber auch klar
verstanden, daß es für die damalige Physik keinen an-
deren Weg gab. Unter den Späteren war es besonders
ERNST
~
I
A
C
H
,
-der diesen Punkt klar ins Licht brachte.

Sowohl die Ableitung als auch die Form der Erhal-
tungssätze werden viel komplizierter, wenn man die
frühere Formulierung der Feldgleichungen zugrunde
legt.

%%%%%%%%%%%%%%%%
\subsection{Allgemeine Bemerkungen}

%%%%%%%%%%%%%%%%

A. Die dargelegte Theorie ist nach meiner Ansicht
die logisch einfachste relativistische Feldtheorie, die
überhaupt möglich ist. Damit ist aber nicht gesagt,
daß die Natur nicht einer komplexeren Feldtheorie
entsprechen könnte. Die Aufstellung komplexerer Feld-
theorien ist vielfach vorgeschlagen worden. Sie lassen
sich betrachten nach folgenden Gesichtspunkten:

D. Man kann gute Argumente dafür anführen, daß
die Realität überhaupt nicht durch ein kontinuierliches
Feld dargestellt werden könne. Aus den Quanten-
phänomenen scheint nämlich mit Sicherheit hervorzu-
gehen, daß ein endliches System von endlicher Energie
durch eine endliche Zahl von Zahlen (Quanten-Zahlen)
vollständig beschrieben werden kann .. Dies scheint zu
einer Kontinuums-Theorie nicht zu passen und muß
zu einem Versuch führen, die Realität durch eine rein
algebraische Theorie zu beschreiben. Niemand sieht
a her, wie die Basis einer solchen Theorie gewonnen
werden könnte.