%
%

%%%%%%%%%%%%%%%%%%%%%%%%%%%%%%%%
\section{Allgemeine Relativitätstheorie}
\label{sec:all-1}
%%%%%%%%%%%%%%%%%%%%%%%%%%%%%%%%

Alle bisherigen Überlegungen beruhen auf der Voraussetzung, daß die Inertialsysteme für die physikalische
Beschreibung gleichberechtigt, den Bezugsräumen von
anderen Bewegungszuständen für die Formulierung der
Naturgesetze aber überlegen seien. Für diese Bevorzugung bestimmter Bewegungszustände vor allen anderen kann gemäß unseren bisherigen Betrachtungen in
den wahrnehmbaren Körpern bzw.\ in dem Begriff der
Bewegung eine Ursache nicht gedacht werden; sie muß
vielmehr auf eine selbständige, d.\ h.\ durch nichts anderes bedingte Eigenschaft des raumzeitlichen Kontinuums
zurückgeführt werden. Insbesondere scheint das Trägheitsgesetz dazu zu zwingen, dem Raum-Zeit-Kontinuum physikalisch-objektive Eigenschaften zuzuschreiben. War es vom Standpunkt \textsc{Newton}s\index{\textsc{Newton}, I.} konsequent
die beiden Begriffe auszusprechen: \enquote{tempus absolutum,
spatium absolutum}, so muß man auf dem Standpunkt
der speziellen Relativitätstheorie von \enquote{continuum absolutum spatii et temporis est} sprechen. Dabei bedeutet
\enquote{absolutum} nicht nur \enquote{physikalisch-real}, sondern
auch \enquote{in ihren physikalischen Eigenschaften selbständig,
physikalisch bedingend, aber selbst nicht bedingt}.