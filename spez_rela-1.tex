%
%

%%%%%%%%%%%%%%%%%%%%%%%%%%%%%%%%
\section{Spezielle Relativitätstheorie}
\label{sec:spe-1}
%%%%%%%%%%%%%%%%%%%%%%%%%%%%%%%%

Die bisherigen Überlegungen sind, abgesehen von der
Voraussetzung der Gültigkeit der EUKLIDischen Geome-
trie, für die Lagerungsmöglichkeiten fester Körper auf
die Voraussetzung gegründet, daß alle Richtungen des
Raumes (bzw. Lagerungen kartesischer Koordinaten-
systeme) physikalisch gleichwertig seien. Es gibt keine
absolute Richtung im Bezugsraume, welche durch ob-
jektive Merkmale ausgezeichnet wäre, sondern nur
Relationen zwischen Richtungen. Man kann diese Aus-
sage als "Relativitätsprinzip in bezug auf die Rich.
tung" bezeichnen, und es wurde gezeigt, daß mittels
des Tensorkalküls diesem Prinzip entsprechend gebaute
Gleichungen (Naturgesetze) gefunden werden können.
Nun stellen wir uns die Frage, ob es auch eine Rela-
tivität hinsichtlich des Bewegungszustandes des Bezugs-
raumes gibt, d. h. ob es relativ zueinander bewegte
Bezugsräume gibt, welche physikalisch gleichwertig
sind. Vom Standpunkt der Mechanik scheinen gleich-
berechtigte Bezugsräume zu existieren. Denn wir mer-
ken beim Experimentieren auf der Erde nichts davon,
daß diese sich mit etwa 30 km/sec Geschwindigkeit um
die Sonne bewegt. Andererseits scheint aber diese
physikalische Gleichwertigkeit nicht für beliebig be-
wegte Bezugsräume zu gelten;' denn die mechanischen
Vorgänge scheinen in bezug auf einen schaukelnden
Eisenbahnwagen nicht nach denselben Gesetzen vor sich
zu gehen, wie in bezug auf einen gleichmäßig fahrenden
EiSenbahnwagen; die Drehung der Erde macht sich
