%
%

%%%%%%%%%%%%%%%%%%%%%%%%%%%%%%%%
\section{Raum und Zeit in der vorrelativistischen Physik}
\label{sec:rau-1}
%%%%%%%%%%%%%%%%%%%%%%%%%%%%%%%%

Die Relativitätstheorie ist aufs engste verbunden mit der Theorie von Raum und Zeit. Deshalb soll mit einer kurzen Untersuchung des Ursprungs unserer Ideen von Raum und Zeit begonnen werden, obwohl ich weiß, daß ich mich dabei auf strittiges Gebiet begebe. Alle Wissenschaft, sei es Naturwissenschaft oder Psychologie, sucht in gewisser Weise unsere Erlebnisse zu ordnen und in ein logisches System zu bringen. Wie hängen die geläufigen Ideen über Raum und Zeit mit dem Charakter unserer Erlebnisse zusammen?

Die Erlebnisse eines Menschen erscheinen uns als in eine Erlebnisreihe eingeordnet, in welcher die einzelnen unserer Erinnerung zugänglichen Einzelerlebnisse nach dem nicht weiter zu analysierenden Kriterium des \enquote{Früher} und \enquote{Später} geordnet erscheinen. Es besteht also für das Individuum eine Ich-Zeit oder subjektive Zeit\index{Zeit!subjektive}. Diese ist an sich nichts Meßbares. Ich kann zwar den Erlebnissen Zahlen zuordnen, derart, daß dem späteren Erlebnis eine größere Zahl zugeordnet wird als dem früheren, aber die Art dieser Zuordnung bleibt zunächst in hohem Maße willkürlich. Ich kann jedoch die Art dieser Zuordnung weiter fixieren durch eine Uhr, indem ich den durch sie vermittelten Erlebnisablauf mit dem Ablauf der übrigen Erlebnisse vergleiche. Unter einer Uhr versteht man ein Ding, welches ab. zählbare Erlebnisse liefert und noch andere Eigenschaften besitzt, von denen im folgenden die Rede sein wird.

Verschiedene Menschen können mit Hilfe der Sprache ihre Erlebnisse bis zu einem gewissen Grade miteinander vergleichen. Dabei zeigt sich, daß gewisse sinnliche Erlebnisse verschiedener Menschen einander entsprechen, während bei anderen ein solches Entsprechen nicht festgestellt werden kann. Jenen sinnlichen Erlebnissen verschiedener Individuen, welche einander entsprechen und demnach in gewissem Sinne überpersönlich sind, wird eine Realität gedanklich zugeordnet. Von ihr, daher mittelbar von der Gesamtheit jener Erlebnisse, handeln die Naturwissenschaften, speziell auch deren elementarste, die Physik. Relativ konstanten Erlebniskomplexen solcher Art entspricht der Begriff des physikalischen Körpers, speziell auch des festen Körpers. Die Uhr ist auch ein Körper bzw.\ ein körperliches System in diesem Sinne. Zum Wesen der Uhr gehört außerdem, daß die an ihr gezählten gleichartigen Teilvorgänge der Erlebnisfolge als einander gleich angesehen werden dürfen.

%%%%%%%%%%%%%%%%%%%%%%%%%%%%%%%%%%%%%%%%%%%%%%%%%%%%%%%%%%%%%%%%

Begriffe und Begriffssysteme erhalten die Berechtigung nur dadurch, daß sie zum Überschauen von Erlebniskomplexen dienen; eine andere Legitimation gibt es für sie nicht. Es ist deshalb nach meiner Überzeugung einer der verderblichsten Taten der Philosophen, daß sie gewisse begriffliche Grundlagen der Naturwissenschaft aus dem der Kontrolle zugänglichen Gebiete des Empirisch-Zweckmäßigen in die unangreifbare Höhe des Denknotwendigen (Apriorischen) versetzt haben. Denn wenn es auch ausgemacht ist, daß die Begriffe nicht aus den Erlebnissen durch Logik (oder sonstwie) abgeleitet werden können, sondern in gewissem Sinn freie Schöpfungen des menschlichen Geistes sind, so sind sie doch ebensowenig unabhängig von der Art der Erlebnisse, wie etwa die Kleider von der Gestalt der menschlichen Leiber. Dies gilt im besonderen auch von unseren Begriffen über Zeit und Raum, welche die Physiker -- von Tatsachen gezwungen -- aus dem Olymp des Apriori herunterholen mußten, um sie reparieren und wieder in einen brauchbaren Zustand setzen zu können.

Wir kommen nun zu den räumlichen Begriffen und Urteilen. Auch hier ist es unerläßlich, die Beziehung der Erlebnisse zu den Begriffen streng ins Auge zu fassen. Auf diesem Gebiete scheint mir \textsc{Poincar\'e}\index{\textsc{Poincar\'e}, H.} die Wahrheit besonders klar erfaßt zu haben in der Darstellung, welche er in seinem Buche: \enquote{La science et l'hypothèse} gegeben hat. Unter allen Veränderungen, welche wir an festen Körpern wahrnehmen, sind diejenigen durch Einfachheit ausgezeichnet, welche durch willkürliche Bewegungen unseres Körpers rückgängig gemacht werden können; \textsc{Poincar\'e}\index{\textsc{Poincar\'e}, H.} nennt sie \enquote{Änderungen der Lage}. Durch bloße Lagenänderungen kann man zwei Körper \enquote{aneinander anlegen}. Das Fundament der Geometrie (Kongruenzsätze) bezieht sich auf die Gesetze, welche jene Lagerungsmöglichkeiten beherrschen. Für den Raumbegriff scheint uns folgendes wesentlich. Man kann durch Anlegen von Körpern $B$, $C \ldots$ an einen Körper $A$ neu Körper bilden, wir wollen sagen, den Körper A fortsetzen. Man kann einen Körper $A$ so fortsetzen, daß er mit jedem anderen Körper $X$ zur Berührung kommt. Wir können den Inbegriff aller Fortsetzungen des Körpers $A$ als den \enquote{Raum des Körpers $A$} bezeichnen. Dann gilt, daß alle Körper sich \enquote{im Raum des (beliebig gewählten) Körpers $A$} befinden. Man kann in diesem Sinne nicht von dem \enquote{Raum} schlechthin, sondern nur von dem \enquote{zu einem Körper $A$ gehörigen Raum} reden. Allerdings spielt im Alltagsleben der Körper Erdkruste eine so dominierende Rolle in der Beurteilung der Lagenverhältnisse der Körper, daß er zu dem ernstlich nicht zu verteidigenden Begriff \emph{des} Raumes (schlechthin) geführt hat. Wir wollen aber, um diesen verhängnisvollen Irrtum auszuschließen, nur von \enquote{Bezugskörper} oder \enquote{Bezugsraum} reden. Erst die allgemeine Relativitätstheorie hat eine Verfeinerung dieses Begriffes nötig gemacht, wie wir später sehen werden.

Ich will nicht näher auf diejenigen Eigenschaften des Bezugsraumes eingehen, welche dazu geführt haben, als Element des Raumes den Punkt einzuführen und den Raum als Kontinuum aufzufassen. Ebensowenig will ich zu analysieren versuchen, durch welche Eigenschaften des Bezugsraumes der Begriff der stetigen Punktreihe oder Linie gerechtfertigt sei. Sind aber diese Begriffe nebst ihrer Beziehung zum festen Körper der Erlebniswelt gegeben, so ist leicht zu sagen, was unter der Dreidimensionalität des Raumes zu verstehen ist, nämlich die Aussage: Jedem Punkt lassen sich drei Zahlen $x_1$, $x_2$ und $x_3$ (Koordinaten) zuordnen, derart, daß diese Zuordnung umkehrbar eindeutig ist, und daß sich $x_1$, $x_2$ und $x_3$ stetig ändern, wenn der zugehörige Punkt eine stetig Punktreihe (Linie) beschreibt.

Die vorrelativistische Physik setzt voraus, daß die Lagerungsgesetze idealer fester Körper der euklidischen Geometrie gemäß seien. Was dies bedeutet, kann z.\ B.\ wie folgt ausgedrückt werden. Zwei an einem festen Körper markierte Punkte bilden eine Strecke. Eine solche kann in mannigfacher Weise gegenüber dem Bezugsraume ruhend gelagert werden. Wenn nun die Punkte dieses Raumes so durch Koordinaten $x_1$, $x_2$, $x_3$ bezeichnet werden können, daß die Koordinatendifferenzen L1x l , L1x 2 , L1x a der Streckenpunkte bei jeder Lagerung der Strecke die nämliche Quadratsumme
\begin{align}
    s^2 = \Delta x_1^2 + \Delta x_2^2 + \Delta x_3^2
\end{align}
liefern, so nennt man den Bezugsraum EUKLIDisch und die Koordinaten kartesische\footnote{Diese Relation D1Uß gelten für beliebige Wahl des Anfangspunktes und der Richtung (Verhältnis L1x]: Lix 2 : .L1x s ) der Strecke.}. Es genügt hierfür sogar, diese Annahme in der Grenze für unendlich kleine Strek. ken zu machen. In dieser Annahme liegen einige weniger spezielle enthalten, auf die wir ihrer grundlegenden Bedeutung wegen aufmerksam machen wollen. Erstens nämlich wird vorausgesetzt, daß man einen idealen festen Körper' beliebig bewegen könne. Zweitens wird vorausgesetzt, daß das Lagerllngsverhalten idealer fester Körper in dem Sinne unabhängig vom Material des Körpers und von seinen Ortsänderungen ist, daß zwei Strecken, welche einmal zur Deckung gebracht werden können, stets und überall zur Deckung gebracht werden können. Diese beiden Voraussetzungen, welche für die Geometrie und überhaupt für die messende Physik von grundlegender Bedeutung sind, entstammen natürlich der Erfahrung; sie beanspruchen in der allgemeinen Relativitätstheorie allerdings nur für (gegenüber astronomischen Dimensionen) unendlich kleine Körper und Bezugsräume Gültigkeit.

%%%%%%%%%%%%%%%%%%%%%%%%%%%%%%%%%%%%%%%%%%%%%%%%%%%%%%%%%%%%%%%%

Die Größe 8 nennen wir die Länge der Strecke. Damit
diese eindeutig bestimmt sei, muß die Länge einer be-
stimmten Strecke willkürlich festgesetzt, z. B. gleich 1
gesetzt werden (Einheitsmaßstab ). Dann sind die Län-
gen aller übrigen Strecken bestimmt. Setzt man die x,
linear abhängig von einem Parameter Ä
\begin{align*}
    x_{\nu} = a_{\nu} + \lambda b_{\nu},
\end{align*}
so erhält man eine Linie, welche alle Eigenschaften der
Geraden der euklidischen Geometrie besitzt. Speziell
folgert man leicht, daß man durch n-maliges Abtragen
einer Strecke 8 auf einer Geraden eine Strecke von der
Länge n · s erhält.
E
i
n
~
Länge bedeutet also das Ergeb-
nis einer längs einer Geraden ausgeführten Messung mit
Hilfe des Einheitsmaßstabes ; sie hat ebenso wie die
gerade Linie.eine vom Koordinatensystem unabhängige
Bedeutung, wie aus dem Folgenden hervorgeht.

Wir kommen nun zu einem Gedankengang, der in
analoger Weise in der speziellen und allgemeinen Rela-
tivitätstheorie eine Rolle spielt. Wir fragen: Gibt es
außer den verwendeten kartesischen Koordinaten noch
andere gleichberechtigte 1 Die Strecke hat eine von
der Koordinatenwahl unabhängige physikalische Be-
deutung, ebenso also auch die Kugelfläche, welche man
erhält als Ort der Endpunkte aller gleichen Strecken,
welche man von einem 'beliebigen Anfangspunkt des
Bezugsraumes aus abträgt. Sind sowohl x. als auch x;
(11 von 1 bis 3) kartesische Koordinaten unseres Bezugs-
raumes, so wird die Kugelfläche in bezug auf jene
beiden Koordinatensysteme durch die Gleichungen aus-
gedrückt:
\begin{align}
    &= \text{konst.}
    \\
    &= \text{konst.}
    \tag{(2a)}
\end{align}
Wie müssen sich die x; aus den Xv ausdrücken, damit die
Gleichungen (2) und (2a) äquivalent seien
~
Denkt man
sich die x; in Funktion der Xv ausgedrückt, so kaDIl man
für genügend kleine L1x" nach dem \textsc{Taylor}schen Satze
setzen:
\begin{align}
    = \ldots
\end{align}
Setzt man dies in (2a) ein und vergleicht mit (1), so sieht
man, daß die x; lineare Gleichungen der Xv sein müssen.
Setzt man demgemäß
\begin{align}
    &=
\end{align}
oder
\begin{align}
    &=
    \tag{(3a)}
\end{align}
so drückt sich die Äquivalenz der Gleichungen (2) und
(2a) in der Form aus
\begin{align}
    \text{($\lambda$ von den unabhängig)}.
    \tag{(3b)}
\end{align}
Hieraus folgt zunächst, daß A eine Konstante sein muß.
Setzt man zunächst Ä == 1, so liefern (2b) und -(3a) die
Bedingungen
\begin{align}
    b_{\nu \alpha} =
\end{align}
wobei $\delta_{\alpha\beta}$ oder ~ ( J ( P == 0 ist, je nachdem LX == ß oder
(X =1= ß. Die Bedingungen (4) heißen Orthogonalitätsbe-
dingungen, die Transformationen (3), (4) lineare ortho-
gonale Transformationen. Verlangt man, daß 8 2 == I.: L1x;
für jedes Koordinatensystem gleich dem Quadrat der
Länge sei und daß stets mit dem gleichen Einheitsmaß-
stabe gemessen werde, so muß Ä == 1 sein. Dann sind
die linearen orthogonalen Transformationen die einzi-
gen, welche den Übergang von einem kartesischen
Koordinatensystem eines Bezugsraumes zu einem ande-
ren vermitteln. Man erkennt, daß bei Anwendung sol-
cher Transformationen die Gleichungen einer Geraden
wieder in die Gleichungen einer Geraden übergehen.
Wir bilden noch die Umkehrung der Gleichungen (3a),
indem wir beiderseits mit b"i multiplizieren und über v
summieren. Man erhält
\begin{align}
    b_{\nu\beta}
\end{align}
Dieselben Koeffizienten b vermitteln also auch die in-
verse Substitution der L1x.,. Geometrisch ist b"or. der
Kosinus des Winkels zwischen der x;-Achse und der
xor.-Achse.

Zusammenfassend können wir sagen: In der eukli-
dischen Geometrie gibt es (in einem gegebenen Bezugs-
raume) bevorzugte Koordinatensysteme, die karte-
sischeri, welche auseinander durch lineare orthogonale
Transformation der Koordinaten hervorgehen. In sol-
chen Koordinaten drückt sich der mit dem Maßstab
meßbare Abstand 8 zweier Punkte des Bezugsraumes in
besonders einfacher Weise aus. Auf diesen Begriff des
Abstandes läßt sich die ganze Geometrie gründen. In
der gegebenen Darstellung bezieht sich die Geometrie
auf wirkliche Dinge (feste Körper), und ihre Sätze sind
Behauptungen über das Verhalten dieser Dinge, welche
zutreffend oder auch unzutreffend sein können.

Gewöhnlich pflegt man die Geometrie so zu lehren,
daß eine Beziehung der Begriffe zu den Erlebnissen
nicht hergestellt wird. Es hat auch Vorteile, dasjenige,
was an ihr rein logisch und von der prinzipiell unvoll-
kommenen Empirie unabhängig ist, zu isolieren. Der
reine Mathematiker kann sich damit begnügen. Er ist
zufrieden, wenn seine Sätze richtig, d. h. ohne logische
Fehler aus den Axiomen abgeleitet sind. Die Frage, ob
die euklidische Geometrie wahr ist oder nicht, hat für
ihn keinen Sinn. Für unseren Zweck aber ist es nötig,
den Grundbegriffen der Geometrie Naturobjekte zuzu ..
ordnen; ohne eine solche Zuordnung ist die Geometrie
für den Physiker gegenstandslos. Für den Physiker
hat es daher wohl einen Sinn, nach der Wahrheit bzw.
dem Zutreffen der geometrischen Sätze zu sprechen.
Daß die so interpretierte euklidische Geometrie nicht
nur Selbstverständliches, d. h. durch Definitionen logisch
Bedingtes ausspricht, erkennt man durch folgende ein-
fache Überlegung, welche von HELMHOLTZ herrührt:

\begin{align}
    V = \iiint d x_1 d x_2 d x_3
\end{align}

\footnote{Es giht also zweierlei kartesische Koordinatensysteme,
welche man als "Rechtssysteme" und "Linkssysteme" be-
zeichnet. Der Unterschied zwischen heiden ist jedem Physiker
und Ingenieur geläufig. Interessant ist, daß man Rechts-
systeme bz,v. Linkssysteme an sich nicht geometrisch defi-
nieren kann, wohl aber die Gegensätzlichkeit beider Typen.}

$\mfraki$ ist ein Vektor, da die Stromdichte definiert ist als
Elektrizitätsdichte, multipliziert mit dem Geschwindig-
keitsvektor der Elektrizität. Also ist es nach den ersten
drei Gleichungen naheliegend, auch e als einen Vektor
zu betrachten. Dann können wir
~
nicht als Vektor
auffassen\footnote{Diese Betrachtungen sollen den Leser mit der Tensor-
betrachtung bekannt machen ohne die besonderen Schwierig-
keiten der vierdimensionalen Betrachtungsweise, damit dann
die entsprechenden Betrachtungen der speziellen Relativitäts-
theorie (\textsc{Minkowski}s Interpl'etation des Feldes) weniger Schwie-
rigkeiten machen.}. Die Gleichungen lassen sich aber leicht
interpretieren, indem man
~
als antisymmetrischen
Tensor vom Range 2 interpretiert. Wir schreiben in
diesem Sinne statt ~ 1 ' ~ 2 ' ~ 3 der Reihe nach ~ 2 3 ' ~ 3 1 ' ~ 1 2 '
Mit Rücksicht auf die Antisymmetrie von ~ I l " können
die ersten drei Gleichungen von (19) und (20) in die
Form gebracht werden
\begin{align}
    &= \frac{1}{c}
    \tag{(19a)} \\
    &= +\frac{1}{c}.
    \tag{(20a)}
\end{align}

~
erscheint demnach im Gegensatz zu e als Größe vom
Symmetrie charakter eines Drehmomentes oder einer
Rotationsgeschwindigkeit. Die Divergenzgleichungen
aber nehmen die Formen an
\begin{align}
    &= \varrho
    \tag{(19b)} \\
    &= 0.
    \tag{(20b)}
\end{align}
Die letzte Gleichung ist eine antisymmetrische Tensor-
gleichung vom dritten Range (die Antisymmetrie der
linken Seite bezüglich jedes Indexpaares ist mit Rück-
sicht auf die Antisymmetrie von $\mfrakh_{\mu\nu}$ leicht zu beweisen).

Sie enthält also trotz ihrer drei Indizes nur eine einzige
Bedingung. Diese Schreibweise ist darum natürlicher
als die übliche, weil sie im Gegensatz zu letzterer ohne
Zeichenänderung auf kartesische Linkssysteme wie auf
Rechtssysteme paßt.