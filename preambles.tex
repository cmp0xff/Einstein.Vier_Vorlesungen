%
%

%TEMPLATE TO GENERATE PDF EASY READABLE ON 6'' EBOOK-READER
%article=mwart;book=mwbk;report=mwrep
%\documentclass[11pt,oneside]{mwbk} %polish style
\documentclass[11pt,oneside]{article} %international style

%adjusting for 6''
\usepackage[
	paperwidth=9cm, 
	paperheight=11.5cm, 
	top=.1cm, 
	left=.1cm, 
	right=.1cm, 
	bottom=.2cm]{geometry}

% other packages (put here what you need)
%\usepackage{polski} % polish package 
%\usepackage[latin2]{inputenc}
%\usepackage[utf8]{inputenc}
\usepackage{graphicx} % images
\usepackage{hyperref} %hyperlinks
\usepackage{calc} % przeliczanie pozycji
\usepackage{amsmath} % matematyczne symbole
%\usepackage{longtable} % dla dlugich tabel
%\usepackage{multirow} % pomoc do tabel

\usepackage{xltxtra}
\defaultfontfeatures{Mapping=tex-text}
\usepackage{polyglossia}
\setdefaultlanguage[
	%spelling=new
	]{german}
\setotherlanguage{french}
\usepackage{unicode-math}

\usepackage{cleveref}

\usepackage[autostyle=true,german=quotes]{csquotes}

% https://tex.stackexchange.com/questions/207532/reset-equation-numbering-after-each-section
\usepackage{chngcntr}

\usepackage{makeidx}

% https://tex.stackexchange.com/questions/1656/footnote-counter-would-like-to-restart-from-1-each-page
\usepackage[perpage]{footmisc}

% https://tex.stackexchange.com/questions/174722/
% add-additional-page-number-from-original-work-in-margin
%\usepackage{marginnote}

% https://tex.stackexchange.com/questions/23601/footnote-number-in-braces-parentheses
\renewcommand{\thefootnote}{\arabic{footnote})}

\makeindex

%FONTS my proposal is TeXGyre
%\usepackage{tgtermes} %times like
%\usepackage{tgcursor} %courier like
%\usepackage{tgschola} %schoolbook like
%\usepackage{tgpagella} %palatin like
%\usepackage{tgchorus} %itc zapf chancery like
%\usepackage{tgbonum} %bookman like
%\usepackage{tgheros} %helvetica like
%\usepackage{tgadventor} %avant garde gothic like

% Differentials
\newcommand{\DD}{\BbbD}
\newcommand{\dd}{d}
\newcommand{\dva}{\mupdelta} % no better way?!
\newcommand{\Dva}{\mupDelta}

	\setmathfont{Latin Modern Math} % default

	\setmathfont{Latin Modern Math}[range={
% 		"00391-"003A9,		% Upright Greek, uppercase
% 		"003B1-"003F5,		% Upright Greek, lowercase
% 		"1D6A8-"1D6E1,	% Bold Greek
		up,	% up everything
	}, math-style=upright]

	\setmathfont{Latin Modern Math}[range={
		%"1D6E2-"1D6FA,		% Italic Greek, uppercase
		%"1D6FC-"1D71B		% Italic Greek, lowercase
		it,	% Italic everything
	}, math-style=ISO]

%	\setmathfont{XITS Math}[range={
%		"1D4B6-"1D4CF,
%		"0212F,"02130, % Script, Latin, lowercase
%	}]

	\setmathfont{Latin Modern Math}[range={
		%"1D608-"1D63B},		% Italic sans serif, Latin
		sfit,	% Italic sans serif
		}, sans-style=italic]

%	\setmathfont{XITS Math}[range={
%		"0212F,				% \mscre
%		"1D4B6-"1D1CF,		% Script, Latin, lowercase: \mscra - \mscrz
%	}]

% 	\setmathfont{Latin Modern Math}[range={
% 		"1D468-"1D49B,		% Bold, Latin: \mbfitA - \mbfitz
% 	}, bold-style=ISO]

% 	\setmathfont{Latin Modern Math}[range={
% 		"1D63C-"1D66F,		% Bold italic sans serif, Latin: \mbfitsansA - 
% 							% \mbfitsansz
% 	}, bold-style=ISO]

% Bracket-like
\newcommand{\rbr}[1]{{\left(#1\right)}}
\newcommand{\sbr}[1]{{\left[#1\right]}}
\newcommand{\cbr}[1]{{\left\{#1\right\}}}
\newcommand{\abr}[1]{{\left<#1\right>}}
\newcommand{\vbr}[1]{{\left|#1\right|}}
\newcommand{\dvbr}[1]{{\left\|#1\right\|}}
\newcommand{\fat}[2]{{\left.#1\right|_{#2}}}

% Fraction-like
\newcommand{\frde}[2]{{\frac{\dd{#1}}{\dd{#2}}}}
\newcommand{\frDe}[2]{{\frac{\DD{#1}}{\DD{#2}}}}
\newcommand{\frpa}[2]{{\frac{\partial{#1}}{\partial{#2}}}}
\newcommand{\frdva}[2]{{\frac{\dva{#1}}{\dva{#2}}}}

% http://unifraktur.sourceforge.net/unifraktur-forum/viewtopic.php?p=718

%IN ADDITION
% on such a small "page", we have to let LaTeX to be sloppy
\frenchspacing 
\sloppy
\widowpenalty=10000 %nie pozostawia wdow i sierot pojedynczych


%LET'S BEGIN

\pagestyle{empty}

%---------
%AUTHOR & TITLE---
\author{Albert Einstein}
\title{Grundzüge der Relativitätstheorie}
%\date{\today}



%-------------------
%ADDITIONAL INFO FOR PDF
%INFO DLA PDF
% \pdfinfo {
% /Author (\@author)
% /Subject (ebook,)
% /Title (\@title)
% /Creator (TexLive&Linux)
% /Producer (pdflatex)
% /CreationDate (D:20121212121212)
% %/ModDate (D:\pdfdate)
% /Keywords (ebook;6inches)
% }

% https://tex.stackexchange.com/questions/332007/how-to-provides-title-and-author-variable-to-hypersetup
\makeatletter
\hypersetup{%
%	xetex,
	unicode		= true,
%	bookmarks	= true,
	linktoc		= section,
	colorlinks	= true,
	pdfauthor	= {\@author},
	pdftitle	= {\@title},
%	pdfsubject	= {The Subject},
%	pdfkeywords	= {Some Keywords},
	pdfproducer	= {XeLaTeX with hyperref},
%	pdfcreator	= {XeLaTeX},
	pdflang		= de-DE,
	}
\makeatother

% https://tex.stackexchange.com/questions/207532/reset-equation-numbering-after-each-section
\counterwithin*{equation}{section}
