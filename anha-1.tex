%
%

%%%%%%%%%%%%%%%%%%%%%%%%%%%%%%%%
\section{%Anhang Ⅰ \
Zum \enquote{kosmologischen Problem}}
\label{sec:anh-1}
%%%%%%%%%%%%%%%%%%%%%%%%%%%%%%%%

Seit dem ersten Erscheinen dieses Büchleins sind
einige Fortschritte der Relativitätstheorie zu verzeich-
nen. Einige davon sollen zunächst kurz erwähnt werden.

Der erste Fortschritt betrifft den überzeugenden
Nachweis von der Existenz der Rot-Verschiebung der
Spektrallinien durch das (negative) Gravitationspotential
des Erzeugungsortes (vgl. S.91). Dieser Nachweis
wurde ermöglicht durch die Entdeckung von sogenann-
ten "Zwergsternen", deren mittlere Dichte die des Was-
sers um einen Faktor von der Größenordnung 10 4 über-
trifft. Für einen solchen Stern (z. B. den lichtschwachen
Begleiter des Sirius), dessen Masse und Radius bestimm-
bar ist\footnote{Die Masse ergibt sich aus der Rückwirkung auf den Sirius
auf spektroskopischem Wege mit Hilfe des NEwToNsehen Ge-
setzes, der Radius aus der absoluten Helligkeit und der aus der
Telnperatur seines Leuchtens erschließbaren Leuchtstärke pro
Flä.cheneinheit.}, ist die nach der Tl1eorie zu erwartende Rot.
versclliebung etwa 20mal so groß wie bei der Sonne und
hat sich tatsächlich in dem erwarteten Betrage nach.
,veisen lassen.

Ein zweiter Fortschritt, der hier kurz erwähnt werden
soll, betrifft das Bewegungsgesetz eines gravitierenden
Körpers. Bei der ursprünglichen Formulierung der
Theorie ,vurde das Bewegungsgesetz für ein gravitie-
rendes Partikel neben den1 Feldgesetz der Gravitation
als eine unabhängige Grundannahme der Theorie ein-
gefüllrt. Vgl. GI. (90); diese spricht aus, daß sich ein
gravitierendes Partikel in einer Geodäte bewegt. Es
ist dies eine hypothetische Übertragung des GALILEI-
sehen Trägheitsgesetzes auf den Fall des Vorhanden seins
"ecllter" Gravitationsfelder. Es hat sich gezeigt, daß
sich dies Bewegungsgesetz - verallgemeinert auf den
Fall beliebig großer gravitierender Massen - aus den
Feldgleichunge11 des leeren Raums erschließen läßt.
Nacll dieser Ableitung wird das Bewegungsgesetz durch
die Bedingung erzwungen, daß das Feld außerhalb der
es erzeugenden Massenpunkte nirgends singulär werden
soll.

Auf einen dritten Fortscllritt, der sich auf das soge-
nannte "kosmologische Problem" bezieht, soll hier aus-
füllrlicher eingegangen werden, teils wegen seiner prin-
zipiellen Bedeutung, teils auch deswegen, weil die Dis-
kussion dieser Fragen noch keineswegs abgeschlossen
ist. Ich fühle mich zu einer genaueren Diskussion auch
dadllrcll gedrängt, daß ich mich des Eindruckes nicht
er,vehren kann, daß bei d.er gegenwärtigen Behandlung
dieses Problems die wichtigsten prinzipiellen Gesichts-
punkte nicllt genügend hervortreten.

Dies Problem läßt sich etwa so forlnulieren. Wir sind
auf Grund der Beobachtungen am Fixstern-Himmel
hinreichend davon überzeugt, daß das System der Fix-
sterne nicht im wesentlichen einer Insel gleicht, die in
einem unendlichen leeren Raum schwebt, daß es also
nicht so etwas gibt wie einen Schwerpunkt der ganzen
in der Welt befindlichen Masse materieller Substanz.
Wir fühlen uns vielmehr zu der Überzeugung gedrängt,
daß es, abgesehen von den lokalen Verdichtungen in
Einzelsterne und Sternsysteme, eine mittlere Dichte der
Materie im Raum gibt, die überall größer als Null ist.
Es entsteht also die Frage: Läßt sich diese von der
Erfahrung nahegelegte Hypothese mit den Gleichungen
der allgemeinen Relativitätstheorie in Einklang bringen 1
Wir haben zuerst das Problem schärfer zu formulieren.
Man denke sich einen Teilraum des Universums, der
eben groß genug ist, daß die mittlere Dichte der in ihm
enthaltenen Stern-Materie als kontinuierliche Funktion
von Xl' .•. , X 4 betrachtet werden kann. In einem sol-
chen Teilraum kann man annähernd ein Inertialsystem
(MINKOwsKI-Raum) finden, auf das man die Stern-
Bewegungen bezieht. Man kann es so einrichten, daß
.die mittlere Geschwindigkeit der Materie in bezug auf
dieses System in allen Koordinatenrichtungen ver-
schwindet. Es bleiben dann noch (nahezu ungeordnete)
Geschwindigkeiten der Sterne übrig, ähnlich der Bewe-
gung der Moleküle eines Gases. Wesentlich ist nun zu-
nächst, daß diese Geschwindigkeiten erfahrungsgemäß
gegen die Lichtgeschwindigkeit sehr klein sind. Es ist
deshalb vernünftig, von der Existenz dieser Relativ-
Bewegungen zunächst ganz abzusehen und die Sterne
ersetzt zu denken durch einen materiellen Staub ohne
(ungeordnete) Relativbewegung der Teilchen gegen-
einander.

Die bisherigen Forderungen genügen aber noch keines-
wegs, um das Problem zu einem hinreichend bestimmten
zu machen. Die einfachste und radikalste Spezialisierung
wäre der Ansatz: die (natürlich gemessene) Dichte e der
Materie ist überall im (vierdimensionalen) Raume die-
selbe, die Metrik ist bei passender Koordinatenwahl
unabhängig von X 4 und
b
e
z
ü
g
l
~
c
h
Xl' XI' X a homogen
und isotrop. Dieser Fall ist es, den ich zunächst als
die natürlichste idealisierte Darstellung für den physi-
kalischen Raum im Großen ansah; er ist auf den Seiten
102-107 dieses Büchleins behandelt. Das Bedenkliche
an dieser Lösung liegt darin, daß man einen negativen
Druck einführen muß, für welchen es keine physika-
lische Rechtfertigung gibt. Ursprünglich hab.e ich zur
Ermöglichung jener Lösung statt des genannten Druckes
eine neues Glied in die Gleichungen eingeführt, welches
vom Standpunkt des Relativitäts-Prinzips erlaubt ist.
Die so erweiterte Gravitationsgleichungen lauten
\begin{align}
    R_{ik} = 0,
\end{align}
wobei A eine universelle Konstante ("kosmologische
Konstante")
b
e
d
e
u
t
e
~
.
Die Einfügung dieses zweiten
Gliedes ist eine Komplizierung der Theorie, welche deren
logische Einfachheit bedenklich vermindert. Seine Ein-
führung kann nur durch die Notlage entschuldigt wer-
den, welche die kaum vermeidbare Einführung einer
endlichen durchschnittlichen Dichte der Materie mit
sich bringt. Beiläufig sei bemerkt, daß in \textsc{Newton}\index{\textsc{Newton}, I.}S
Theorie dieselbe Schwierigkeit besteht.

Aus diesem Dilemma hat der Mathematiker FRIED-
MANN einen Ausweg gefunden\footnote{Er hat gezeigt, daß es nach den Feldgleichungen möglich
ist, eine endliche Dichte im ganzen Raume (dreidimensional
aufgefaßt) zu haben, ohne die Feldgleichungen ad hoc zu er-
weitern. Zeitschr. f. Physik 10 (1922).}. Sein Ergebnis hat
dann durch \textsc{Hubble}S Entdeckung der Expansion des
Fixstern-Systems (mit der. Distanz gleichmäßig an-
wachsender Rot-Verschiebung der Spektrallinien) eine
überraschende Bestätigung gefunden. Das Folgende ist
im wesentlichen nichts anderes als eine Darlegung von
IfRIEDMANNS Idee: Vierdimensionaler Raum, der bezüg-
lich dreier Dimensionen isotrop ist.

Wir nehmen wahr, daß die Sternsysteme von uns
aus gesehen nach allen Richtungen hin ungefähr gleicll
dicht verteilt sind. Wir sehen uns dadurch zu der An-
nahme gedrängt, daß diese räumliche Isotropie des
Systems für alle Beobachter zutreffen würde, für jeden
Ort und jede Zeit eines gegen die ihn umgebende Ma-
terie ruhenden Beobachters. Dagegen machen wir nicht
l11ehr die Anllahme, daß die mittlere Dichte der Materie
für einen relativ z'ur benachbarten Materie ruhenden
Beobachter zeitlich konstant sei. Damit entfällt auch
die Annahme, daß der Ausdruck des metrischen Feldes
die Zeit nicht enthalte.

Wir müssen nun eine mathematische Form finden
für die Voraussetzung, daß die Welt in räumlicher Be-
ziehung allenthalben isotrop sei. Durch jeden Punkt P
des (vierdimensionalen) Raumes geht eine Teilchen-
Bahn (im folgenden kurz "Geodäte" genannt). P und
Q seien zwei infinitesimal benachbarte Punl\:te einer
solchen Geodäte. Dann werden wir zu verlangen haben, daß bezüglich jeder \enquote{Drehung} des Koordinatensystems
um Pund Q der Ausdruck des Feldes invariant sein
soll. Dies soll gelten für jedes Element jeder Geodäte.

Diese Forderung beschränkt nicht nur die Metrik,
sondern auch die Koordinatenwahl, von welch letzterer
Beschränkung wir uns nach Auffindung der Metriken
von dem verlangten Symmetrie-Charakter wieder frei
machen können.

Die Forderung einer solchen Invarianz verlangt, daß
die Geodäte in ihrem ganzen Verlauf der Drehungsachse
angehört und all ihre Punkte bei der Drehung des
Koordinatensystems fest bleiben. Die Lösung soll also
drehungsinvariant sein bezüglich aller Drehungen des
Koordinatensystems um alle die dreifach unendlich
vielen Geodäten.

Auf die deduktive Ableitung der Lösung dieses Pro-
blems will ich hier der Kürze halber nicht eingehen.
für einen dreidimel1sionalen Raum erscheint es jedoch
anschaulich evident, daß eine bezüglich zweifach un-
endlich vielen Linien drehungsinvariante Metrik im
,vesentlichen (Ien Typus einer (bei passender Koordi-
natenwahl) zentralsymmetrischen Lösung haben muß,
wobei die Drehachsen die radial verlaufenden Geraden
sind, die ja aus Symmetriegründen Geodäten sind. Die
Flächen konstanten Radius sind dann Flächen kon-
stantel (positiver) Krümmung, welche auf den (radialen)
Geodäten überall senkrecht stehen. In invarianter Aus-
drucksweise ergibt sich also:

Es gibt eine zu den Geodäten orthogonale Flächen-
schar. Jede dieser Flächen ist eine Fläche konstanter
Krümmung. Je zwei Flächen dieser Schar schneiden
aus diesen Geodäten gleich, lange Stücke heraus.

Bemerkung. Der so anscllaulich gewonnene Fall
ist nur insofern nicht der allgemeine, als die Flächen
der Schar auch Flächen negativer konstanter Krüm-
mung oder EUKLIDisch(verschwindende Krümmung) sein
können.

Indem uns interessierenden vierdimensionalen Fall
ist es genau analog. Es ist ferner kein wesentlicher
Unterscllied, ,venn der metrische Raum vom Trägheits-
index 1 ist; nur muß man die radialen Richtungen zeit-
artig, die in den Flächen der Schar liegenden Richtungen
dementsprechend raumartig wählen. Die Achsen der
lokalen Liclltkegel aller Punkte liegen auf den radialen
Linien.

%%%%%%%%%%%%%%%%
\subsection{Koordinatenwabl}

%%%%%%%%%%%%%%%%

Statt jener vier Koordinaten, für welche die räum-
liche Isotropie des Kontinuums am unmittelbarsten
hervortritt, wählen wir nun andere Koordinaten, die
vom Standpunkt der physikalischen Interpretation be-
quelner sind.
Als zeitartige Linien, auf denen Xl' X 2 , x 3 konstant
sind und X 4 allein variabel, wählen wir die Teilchen-Geodäten, welche in der zentralsymmetrischen Dar-
stellung die vom Zentrum ausgellenden Geraden sind.
x 4 sei ferner gleich dem metrischen Abstand vom Zen-
trum. In solchen Koordinaten ausgedrückt, ist die
Metrik von der speziellen Gestalt

%%%%%%%%%%%%%%%%
\subsection{Die Feldgleichungen}

%%%%%%%%%%%%%%%%

Wir haben nun ferner den Feldgleichungen der Gra-
vitation Genüge zu leisten, und zwar den Feldgleichun-
gen ohne das früher ad hoc eingeführte "kosmologische
Glied":

%%%%%%%%%%%%%%%%
\subsection{Der Spezialfall verschwindender räumlicher
Krümmung ($z = 0$)}

%%%%%%%%%%%%%%%%

Der einfachste Sonderfall bei nicht verschwindender
Dichte e ist der Fall

%%%%%%%%%%%%%%%%
\subsection{Lösungen der Gleichung'en
im Falle nicht verschwindender riiumlicher Krümmung}

%%%%%%%%%%%%%%%%

Berücksichtigt man eine räunlliche Krümmung
räumlichen Schnittes (x 4 = konst.), so hat man
Gleichungen

%%%%%%%%%%%%%%%%
\subsection{Erweiterung der vorstehenden Überlegungen
durch Verallgemeinerung des Ansatzes bezüglich
der ponderabeln Materie}

%%%%%%%%%%%%%%%%

Bei allen bisher erlallgten Lösungen gibt es eillen
Zustand des Systems, in welchem die Metrik singulär

%%%%%%%%%%%%%%%%
\subsection{\enquote{Teilchen-Gas}, nach der speziellen Relativtätstheorie behandelt}

%%%%%%%%%%%%%%%%

Wir denken uns einen Schwarm parallel bewegter
Teilchell von der Masse m. Er kann auf Rulle trans-
formiert werden, die räumliche Dichte der Teilchen, (1,
hat dann \textsc{Lorentz}-invariante Bedeutung. Auf ein be-
liebiges \textsc{Lorentz}-System bezogen, hat dann


%%%%%%%%%%%%%%%%
\subsection{Zusammenfassende und sonstige Bemerkungen}\footnote{Vgl.\ Anmerkung am Schluß dieses Anhangs.}

%%%%%%%%%%%%%%%%

e Gravitationsgleichullgen ist zwar relativistisch mög-
lich, vom Standpunkt der logischen Ökonomie aber ver-
werflich. Wie \textsc{Friedmann}\index{\textsc{Friedmann}, A.} zuerst gezeigt hat, kann
man eine allenthalben endliche Dichte der Materie mit
der ursprünglichen Form der Gravitationsgleichungen
in Einklang bringen, wenn man die zeitliclle Veränderlicllkeit des metriscllell Abstandes distanter Massenpunkte zuläßt\footnote{'Vürde die \textsc{Hubble}-Expansion bei Aufstellung der allgemeinen Relativitätstheorie bereits entdeckt gewesen sein, so
wäre es nie zur Einführung des kosmologischen Gliedes ge-
konlmen. Es erscheint nun aposteriori um so ungerechtfertigter,
ein solches Glied in die Feldgleichungen einzuführen, als dessen
Einführung seine einzige ursprüngliche Existenzberechtigung
- zu einer natürlichen Lösung des kosmologischen Problems
zu führen - einbüßt.}.